%% LyX 2.2.2 created this file.  For more info, see http://www.lyx.org/.
%% Do not edit unless you really know what you are doing.
\documentclass[oneside,english]{book}
\usepackage[latin9]{inputenc}
\setcounter{secnumdepth}{3}
\setcounter{tocdepth}{3}
\usepackage{amssymb}
\usepackage{makeidx}
\makeindex
\usepackage{babel}
\begin{document}

\title{Open Quantum Systems Theory behind Quanta$\rho\varepsilon\iota$
Package}

\author{Tom\'{a}\v{s} Man\v{c}al}

\date{\today}

\maketitle
\tableofcontents{}

\chapter{Introduction}

This text presents a theoretical backbround for a Python software
package Quanta$\rho\varepsilon\iota$ (see \verb$http://github.com/tmancal74/quantarhei$).

Quanta$\rho\varepsilon\iota$ deals with the dynamics and spectroscopy
of molecular systems. This is said mainly to stress that this package
is not intended to treat the part of physics usually know as solid
state physics or solid state theory. There is no prejudice or personal
(dis)liking behind this. It is to say that there are huge differences
in theoretical approaches to a problem of regular solid and that of
a small molecular aggregate. Many concepts discussed here do apply
in solid state problems, and many do not. It is a huge mistake to
think that concepts that bare the same name in two different branches
of physics are automatically transferable between the two theoretical
domains. This is definitely the case for fields such as chemical physics
and solid state physics.

In this text, we do not explain quantum mechanics itself. It is expected
that the reader has a knowledge of quantum mechanics on the level
of a completed undergraduate course. Any standard textbook of quantum
mechanics will suffice for a start. This assumpltion will allow us
to skip elementary introduction into concepts of quantum mechanics
and we could concentrate on the question how these concepts are represented
in Quanta$\rho\varepsilon\iota$.

The following books should be consulted to get a fuller picture, especially
if you want to use Quanta$\rho\varepsilon\iota$ before this text
grows into a self-contained informal textbook: 
\begin{itemize}
\item Volkhard May and Oliver K�hn, \emph{Charge and Energy Transfer in
Molecular Systems}, Wiley-VCH, Berlin, 2000 (and later editions)
\item Shaul Mukamel, \emph{Principles of Nonlinear Spectroscopy}, Oxford
University Press, Oxford, 1995
\item Leonas Valkunas, Darius Abramavicius and Tom\'{a}\v{s} Man\v{c}al,
\emph{Molecular Excitation Dynamics and Relaxation}, Wiley-VCH, Weinheim,
2013. 
\end{itemize}

\chapter{Closed Quantum Systems}

\paragraph{Concepts introduced in this chapter}
\begin{itemize}
\item Closed quantum system
\item State vector
\item Hamiltonian
\item Schroedinger equation, evolution operator
\item Hilbert space, representation in basis states
\item Systems with many degrees of freedom
\end{itemize}
Quantum theory is build upon a concept of a closed quantum system\index{closed quantum system}.
This is a necessary idealization which we need to start with. For
the sake of interpretation of quantum mechanics, it would be better
to completely forget about such an idealization once the main concepts
of quantum theory become somewhat more familiar. All quantum systems,
at the latest at the moment they are measured, are open quantum systems,
and the only closed system is the whole universe.

\section{Time Evolution of a State Vector}

We describe those physical properties of a molecular (or atomic) system
that are important for its dynamics by the so-called Hamiltonian\index{Hamiltonian}
(or energy) operator $\hat{H}$. The state of the system is described
by a state vector\index{state vector} $|\psi(t)\rangle$ which can
be time dependent. All operators, not just the Hamiltonian, can act
on the state vector and change it
\begin{equation}
|\phi\rangle=\hat{H}|\psi\rangle.\label{eq:H_on_psi}
\end{equation}
Eq. (\ref{eq:H_on_psi}) is not yet meant to describe any physics.
We mean acting in a matematical sense. We have a vector $|\psi\rangle,$
and when we apply an operator $\hat{H}$ to it, we get a different
vector. 

The signature of two vectors being different is the fact that their
mutual scalar product, once they are normalized, is not one. For all
state vectors we assume
\begin{equation}
\langle\psi(t)|\psi(t)\rangle=1,
\end{equation}
i.e. we assume that they are normalized to one. In general, if $|\psi\rangle$
and $|\phi\rangle$ are two normalized state vectors, then
\begin{equation}
\langle\psi|\phi\rangle\leq1,
\end{equation}
and the equality applies when the vectors are the same.

In Quanta$\rho\varepsilon\iota$ the state vector is represented by
the \verb$StateVector$\index{quantarhei.StateVector class} class,
and the Hamiltonian is correspondingly represented by the \verb$Hamiltonian$\index{quantarhei.Hamiltonian class}
class (see Quanta$\rho\varepsilon\iota$ documentation). 

The time evolution of a closed quantum mechanical system is governed
by the \index{Schroedinger equation}Schr\"{o}dinger equation which
reads
\begin{equation}
\frac{\partial}{\partial t}|\psi(t)\rangle=-\frac{i}{\hbar}\hat{H}|\psi(t)\rangle.
\end{equation}
Quanta$\rho\varepsilon\iota$ has a class \verb$StateVectorEvolution$\index{quantarhei.StateVectorEvolution class}
which represents $|\psi(t)\rangle$ on a given interval of time with
a given initial state $|\psi_{0}\rangle$. 

A very important quantity for formal manipulations in quantum mechanics
is the so-called evolution operator\index{evolution operator}
\begin{equation}
U(t,t_{0})=\exp\left\{ -\frac{i}{\hbar}\hat{H}(t-t_{0})\right\} .
\end{equation}
If at time $t=t_{0}$ we have $|\psi(t_{0})\rangle=|\psi_{0}\rangle$
then 
\begin{equation}
|\psi(t)\rangle=U(t,t_{0})|\psi_{0}\rangle.
\end{equation}

From the point of view of Quanta$\rho\varepsilon\iota$, the state
vector $|\psi\rangle$ is a complex vector on a vector space with
some limited number of dimensions. Similarly, operators such as $\hat{H}$
are represented by matrices. The raw vectors corresponding to the
$|\psi\rangle$ and raw matrices corresponding to $\hat{H}$ can be
accessed trough the atribute \verb$data$ of the corresponding object
in Quanta$\rho\varepsilon\iota$. Time dependent quantites have one
additional index specifying time. 

\section{Molecular Hamiltonian}

We assume that molecules can be to a certain extent treated within
the Born-Oppenheimer approximation. This means that there are some
distinguisheble electronic states in which we can find the molecule,
and that are to some extent independent of the motional states of
the nuclei which form the backbone of the molecule. We will assume
that a given molecule has an electronic ground state $|g\rangle$
and an excited state $|e\rangle$ with certain energies $\epsilon_{g}$
and $\epsilon_{e}$, respectively. States $|g\rangle$ and $|e\rangle$
are eigenstates of the molecule electronic Hamiltonian and correspondingly,
one can write
\begin{equation}
\hat{H}_{{\rm mol}}=\epsilon_{g}|g\rangle\langle g|+\epsilon_{e}|e\rangle\langle e|.\label{eq:bare_2level}
\end{equation}
This is the simplest bare molecule of interest in Quanta$\rho\varepsilon\iota$. 

\subsection{Molecule with a Single Vibrational Mode}

To add a single additional degree of freedom (DOF) to the system we
have to extend the working \index{Hilbert space}Hilbert space. The
Hamiltonian, eq. (\ref{eq:bare_2level}) describes a single DOF and
the corresponding Hilbert space is a vector space defined by all possible
linear combinations of vectors $|g\rangle$ and $|e\rangle.$ Let
us imagine that we deal with a diatomic molecule which oscillates
with a characteristic frequency $\omega$. Let us assume that the
oscillations proceed irrespective of the electronic state in which
the molecule finds itself, or in otherwords, let us assume that the
oscillatory DOF of the molecule does not interact with the electronic
DOF of the molecule. One can write the Hamiltonian as a sum of the
vibrational and electronic parts
\begin{equation}
\hat{H}_{{\rm mol}}=\hat{H}_{{\rm vib}}+\hat{H}_{{\rm el}}=\sum_{n=0}^{\infty}n\hbar\omega|n\rangle\langle n|+\sum_{k=g,e}\epsilon_{k}|k\rangle\langle k|.\label{eq:Ham_el_1vib}
\end{equation}
We disregarded the zero point energy of the oscillator as this is
only an indignificant shift of the total energy. 

The system is now described two sets of states. One is the set of
electronic states $|k\rangle$, $k=g,e$ and one is the set of vibrational
states $|n\rangle$, $n=1,2,\dots$. To precisely describe the set
we have to specify states of both of the components. Possible states
of the system are therefore e.g. $|g\rangle|1\rangle$ or $|e\rangle|20\rangle$
or any of the linear combinations of such states. Correspondingly,
the basis of the common Hilbertspace describing both components of
the system has to be composed of the product states $|k\rangle|n\rangle$.
The components of the Hamiltonian, Eq. (\ref{eq:Ham_el_1vib}) are
not defined on a Hilbert space defined by the product states, but
can be easily made so by just assuming that the empty space next to
them holds a unity operator on the Hilbert space which we are missing.
One can for instance write
\begin{equation}
\sum_{n=0}^{\infty}n\hbar\omega|n\rangle\langle n|=\sum_{n=0}^{\infty}n\hbar\omega|n\rangle\langle n|\otimes\hat{1}_{{\rm el}}=\sum_{n=0}^{\infty}n\hbar\omega|n\rangle\langle n|\sum_{k=e,g}|k\rangle\langle k|,\label{eq:intro_1el}
\end{equation}
where $\hat{1}_{{\rm el}}$ is the unity operator on the electronic
Hilbert space expressed in the later part of the Eq. (\ref{eq:intro_1el})
using the completeness relation
\begin{equation}
\hat{1}_{{\rm el}}=\sum_{k=g,e}|k\rangle\langle k|.
\end{equation}
Similarly, we have 
\begin{equation}
\sum_{k=g,e}\epsilon_{k}|k\rangle\langle k|=\sum_{k=g,e}\epsilon_{k}|k\rangle\langle k|\otimes\hat{1}_{{\rm vib}}=\sum_{k=g,e}\epsilon_{k}|k\rangle\langle k|\sum_{n}|n\rangle\langle n|.
\end{equation}
The definitions we introduced here allow us for instance to rearange
the Hamiltonian into a form
\begin{equation}
\hat{H}_{{\rm mol}}=\left(\epsilon_{g}+\sum_{n}n\hbar\omega|n\rangle\langle n|\right)|g\rangle\langle g|+\left(\epsilon_{e}+\sum_{n}n\hbar\omega|n\rangle\langle n|\right)|e\rangle\langle e|.\label{eq:Hem_el_1vib_equiv}
\end{equation}

If we do not insist on writing out the vibrational Hamiltonian using
the vibrational eigenstates, we can as well write
\begin{equation}
\sum_{n}n\hbar\omega|n\rangle\langle n|=\frac{\hat{p}^{2}}{2m}+\frac{m\omega^{2}}{2}\hat{q}^{2}.
\end{equation}
This form of the vibrational Hamiltonian will enable us to introduce
interaction between the two DOF is a very sensible manner.

It is easy to see that if we prepare the molecular system described
by the Hamiltonian, Eq. (\ref{eq:Ham_el_1vib}) or equivalently Eq.
(\ref{eq:Hem_el_1vib_equiv}), in its excited state, it will stay
there indefinetly. The state vector $|\psi\rangle$ can be 

\section{Systems of More than One Molecule}
\begin{verbatim}

\end{verbatim}

\chapter{Bath Correlation Functions and Spectral Densities}

\section{Bath Correlation Function}

Bath correlation function is defined as a two point correlation function
of the bath part $\Delta V$ of the system-bath interaction operator,
i.e. as

\begin{equation}
C(t)=\frac{1}{\hbar^{2}}{\rm {\rm Tr}}_{B}\{U_{B}^{\dagger}(t)\Delta VU_{B}(t)\Delta Vw_{{\rm eq}}\}
\end{equation}
where $w_{{\rm eq}}$ is the equilibrium bath density operator, $U_{B}(t)$
is the bath evolution operator and the trace is taken over the bath
degrees of freedom. The bath correlation function is a complex quantity
and as such it has a real part

\begin{equation}
C^{\prime}(t)=\frac{1}{2}\left[C(t)+C^{*}(t)\right]
\end{equation}
and an imaginary part
\begin{equation}
C^{\prime\prime}(t)=-\frac{i}{2}\left[C(t)-C^{*}(t)\right]
\end{equation}
so that
\begin{equation}
C(t)=C^{\prime}(t)+iC^{\prime\prime}(t)
\end{equation}


\section{Spectral Density}

A very important quantity is the Fourier transform of the bath correlation
function

\begin{equation}
\tilde{C}(\omega)=\int\limits _{0}^{\infty}{\rm d}t\ C(t)e^{i\omega t}=2{\rm Re}\int\limits _{0}^{\infty}{\rm d}t\ C(t)e^{i\omega t}
\end{equation}
It is sometimes referred to as \index{spectral density}\emph{spectral
density}, but we will reserve this name for a different quantity.
We will follow Ref. {[}Mukamel1995{]}. The Fourier transform $\tilde{C}(\omega)$
can be split into even and odd parts defined as
\begin{equation}
\tilde{C}^{\prime}(\omega)=\int\limits _{-\infty}^{\infty}{\rm d}t\ C^{\prime}(t)e^{i\omega t},\;\tilde{C}^{\prime\prime}(\omega)=i\int\limits _{-\infty}^{\infty}{\rm d}t\ C^{\prime\prime}(t)e^{i\omega t}
\end{equation}
so that 
\begin{equation}
\tilde{C}(\omega)=\tilde{C}^{\prime}(\omega)+\tilde{C}^{\prime\prime}(\omega)
\end{equation}
It can be shown (see {[}Mukamel1995{]}) that 
\begin{equation}
\tilde{C}(-\omega)=e^{-\frac{\hbar\omega}{k_{B}T}}\tilde{C}(\omega)\label{eq:C_omega_symmetry}
\end{equation}
 and
\begin{equation}
\tilde{C}(\omega)=\left[1+\coth(\hbar\omega/2k_{B}T)\right]\tilde{C}^{\prime\prime}(\omega)\label{eq:C_omega_by_C_2primes}
\end{equation}
Due to the relation between positive and negative frequency values
of the Fourier transform of the bath correlation function, we can
define it completely through the odd function $\tilde{C}^{\prime\prime}(\omega)$
which is a Fourier transform of the imaginary part of the correlation
function.

Spectral density

\begin{equation}
J(\omega)=\sum_{\xi}|g_{\xi}|^{2}\delta(\omega-\omega_{\xi})
\end{equation}


\chapter{Standard Redfield Theory}

In this section we define the \index{standard Redfield relaxation tensor}standard
Redfield relaxation tensor in a form which is implemented in Quanta$\rho\epsilon\iota$.
Our formulation is based on the reference {[}MayKuehn{]}

\section{General Formula}

\begin{equation}
K_{n}^{(I)}(-\tau)=U_{S}(\tau)K_{n}U_{S}^{\dagger}(\tau)
\end{equation}

\begin{equation}
\Lambda_{m}=\sum_{n}\int\limits _{0}^{\infty}{\rm d}\tau\ C_{mn}(\tau)K_{n}^{(I)}(-\tau)
\end{equation}

\begin{equation}
\frac{\partial}{\partial t}\rho(t)=-\frac{i}{\hbar}\left[H_{S},\rho(t)\right]_{-}-{\cal R}\rho(t)
\end{equation}

\[
{\cal R}\rho(t)=-\frac{1}{\hbar^{2}}\sum_{m}\left(K_{m}\Lambda_{m}\rho(t)+\rho(t)\Lambda_{n}^{\dagger}K_{m}\right)
\]
 

\begin{equation}
+\frac{1}{\hbar^{2}}\sum_{m}\left(K_{m}\rho(t)\Lambda_{m}^{\dagger}+\Lambda_{m}\rho(t)K_{m}\right)
\end{equation}


\subsection{Relaxation rate}

In a representation of Hamiltonian eigenstates $|a\rangle$ we can
write e.g.
\[
\frac{\partial}{\partial t}\rho_{aa}(t)=\sum_{b}\left(\frac{1}{\hbar^{2}}\sum_{m}\langle a|K_{m}|b\rangle\rho_{bb}(t)\langle b|\Lambda_{m}^{\dagger}|a\rangle\right)
\]
\begin{equation}
+\sum_{b}\left(\frac{1}{\hbar^{2}}\sum_{m}\langle a|\Lambda_{m}|b\rangle\rho_{bb}(t)\langle b|K_{m}|a\rangle\right),
\end{equation}
where we ignored the so-called non-secular terms which connect oscillating
coherence elements $\rho_{ab}$ of the density matrix with the populations
$\rho_{aa}$. From the definition of $\Lambda_{m}$ we have
\begin{equation}
\langle a|\Lambda_{m}|b\rangle=\sum_{n}\int\limits _{0}^{\infty}{\rm d}\tau\ C_{mn}(\tau)e^{-i\omega_{ab}\tau}\langle a|K_{n}|b\rangle=\sum_{n}\bar{C}_{mn}(\omega_{ba})\langle a|K_{n}|b\rangle,
\end{equation}
where we used that $U_{S}(t)=\sum_{a}e^{-\omega_{a}t}|a\rangle\langle a|$
with $\omega_{a}=\epsilon_{a}/\hbar$ and $\epsilon_{a}$ the eigenergy
of the system corresponding to state $|a\rangle$. For the element
$\langle a|\Lambda_{m}^{\dagger}|b\rangle$ we have $\langle a|\Lambda_{m}^{\dagger}|b\rangle=\langle b|\Lambda_{m}|a\rangle$
and correspondingly the rate of transfer from $|b\rangle$ to $|a\rangle$
is 
\[
\hbar^{2}K_{ab}=\sum_{mn}\bar{C}_{mn}(\omega_{ba})\langle a|K_{n}|b\rangle\langle b|K_{m}|a\rangle+\bar{C}_{mn}^{*}(\omega_{ba})\langle a|K_{m}|b\rangle\langle a|K_{m}|b\rangle
\]
\begin{equation}
\hbar^{2}K_{ab}=\sum_{mn}2{\rm Re}\bar{C}_{mn}(\omega_{ba})\langle a|K_{n}|b\rangle\langle b|K_{m}|a\rangle.
\end{equation}
We can see that the integral above is a half of the Fourier transform
and that there is a relation between the $\bar{C}(\omega)=\int\limits _{0}^{\infty}{\rm d}\tau\ C(\tau)e^{i\omega\tau}$
and the full Fourier transform $\tilde{C}(\omega)=\int\limits _{-\infty}^{\infty}{\rm d}\tau\ C(\tau)e^{i\omega\tau}$
because the correlation function is symmetric
\begin{equation}
C(-t)=C^{*}(t).
\end{equation}
We can show that
\begin{equation}
2{\rm Re}\bar{C}(\omega)=\tilde{C}(\omega),
\end{equation}
and correspondingly
\begin{equation}
\hbar^{2}K_{ab}=\sum_{mn}\tilde{C}_{mn}(\omega_{ba})\langle a|K_{n}|b\rangle\langle b|K_{m}|a\rangle.
\end{equation}


\section{Analytical Results}

\subsection{Homo-dimer}

The system-bath interaction operators are
\begin{equation}
K_{m}=|m\rangle\langle m|
\end{equation}

If both energy gap fluctuate with the same energy gap correlation
function $C(t)$, and the fluctuations on individual sites are independent
from each other ($C_{mn}(t)=\delta_{mn}C(t))$, the rate $K_{ab}$
of the energy transfer from state $|b\rangle$ to state $|a\rangle$
reads as (we set $\hbar=1$)
\begin{equation}
K_{ab}=\sum_{n}|\langle a|n\rangle|^{2}|\langle b|n\rangle|^{2}\tilde{C}(\omega_{ba}).
\end{equation}
For a homodimer all coefficients $|\langle b|n\rangle|^{2}=\frac{1}{2}$
and the sum over $n$ gives two contributions which are exactly the
same. This means
\begin{equation}
\sum_{n}|\langle a|n\rangle|^{2}|\langle b|n\rangle|^{2}=\sum_{n=1}^{2}\frac{1}{4}=\frac{1}{2}.
\end{equation}
For an overdumped Brownian oscillator spectral density
\begin{equation}
\tilde{C}^{\prime\prime}(\omega)=\frac{2\lambda\omega\left(\frac{1}{\tau_{c}}\right)}{\omega^{2}+\left(\frac{1}{\tau_{c}}\right)^{2}}
\end{equation}
 we have

\begin{equation}
K_{ab}=\frac{1}{2}\tilde{C}(\omega_{ba})=\frac{1}{2}\left(1+\coth\left(\frac{\omega_{ba}}{2k_{B}T}\right)\right)\frac{2\lambda\omega_{ba}\left(\frac{1}{\tau_{c}}\right)}{\omega_{ba}^{2}+\left(\frac{1}{\tau_{c}}\right)^{2}}.
\end{equation}
Given that for a homodimer $\omega_{12}=2J$ we have
\begin{equation}
K_{12}=\frac{1}{2}\tilde{C}(\omega_{21})=\frac{\lambda J}{2}\frac{1+\coth\left(\frac{J}{k_{B}T}\right)}{J^{2}\tau_{c}+\frac{\hbar^{2}}{4\tau_{c}}}.
\end{equation}

This formula is used to test the calculations of Redfield rates and
of the Redfield tensor in Quanta$\rho\epsilon\iota$.

\subsection{Hetero-dimer}

\chapter{Standard F�rster Theory}

\begin{equation}
K_{AD}=|J|^{2}2{\rm Re}\int\limits _{0}^{\infty}{\rm d}t\ \underbrace{e^{-g_{A}(t)-i\omega_{A}t}}_{A(t)}e^{i\omega t}\underbrace{e^{-g_{D}^{*}(t)+i(\omega_{D}-2\lambda_{D})t}}_{F(t)}e^{i\omega t}.
\end{equation}
Absorption spectrum

\begin{equation}
\alpha(\omega)\approx\int\limits _{-\infty}^{\infty}{\rm d}t\ A(t)e^{i\omega t}=A(\omega)
\end{equation}
Fluorescence spectrum

\begin{equation}
f(\omega)\approx\int\limits _{-\infty}^{\infty}{\rm d}t\ F(t)e^{i\omega t}=F(\omega)
\end{equation}
Fourier transform

\begin{equation}
A(t)=\frac{1}{2\pi}\int\limits _{-\infty}^{\infty}{\rm d}\omega\ A(\omega)e^{-i\omega t}
\end{equation}

\appendix

\chapter{Microscopic Derivation of Spectral Density Symmetries}

Here we will derive the Eqs. (\ref{eq:C_omega_symmetry}) and (\ref{eq:C_omega_by_C_2primes}).

\printindex{}
\end{document}
