%% LyX 2.1.4 created this file.  For more info, see http://www.lyx.org/.
%% Do not edit unless you really know what you are doing.
\documentclass[english]{revtex4-1}
\usepackage[T1]{fontenc}
\usepackage[latin9]{inputenc}
\setcounter{secnumdepth}{3}
\usepackage{amssymb}
\usepackage{esint}
\usepackage{babel}
\begin{document}

\title{Open Quantum Systems Theory behind Quantarhei Package}


\author{Tom\'{a}\v{s} Man\v{c}al}


\affiliation{Faculty of Mathematics and Physics, Charles University, Ke Karlovu
5, 121 16 Prague 2, Czech Republic}


\date{\today}
\begin{abstract}
In this document, we summarize the theory of open quantum systems
as it is implemented in the Quantarhei package. Before it grows into
a self-contained text, the following books should be consulted to
gat a full picture: Volkhard May and Oliver K�hn, \emph{Charge and
Energy Transfer in Molecular Systems}, Wiley-VCH, Berlin, 2000 (and
later editions), Shaul Mukamel, \emph{Principles of Nonlinear Spectroscopy},
Oxford University Press, Oxford, 1995 and Leonas Valkunas, Darius
Abramavicius and Tom\'{a}\v{s} Man\v{c}al, \emph{Molecular Excitation
Dynamics and Relaxation}, Wiley-VCH, Weinheim, 2013. 
\end{abstract}
\maketitle
\tableofcontents{}


\section{Bath Correlation Functions and Spectral Densities}


\subsection{Bath Correlation Function}

Bath correlation function is defined as a two point correlation function
of the bath part $\Delta V$ of the system-bath interaction operator,
i.e. as

\begin{equation}
C(t)=\frac{1}{\hbar^{2}}{\rm {\rm Tr}}_{B}\{U_{B}^{\dagger}(t)\Delta VU_{B}(t)\Delta Vw_{{\rm eq}}\}
\end{equation}
where $w_{{\rm eq}}$ is the equilibrium bath density operator, $U_{B}(t)$
is the bath evolution operator and the trace is taken over the bath
degrees of freedom. The bath correlation function is a complex quantity
and as such it has a real part

\begin{equation}
C^{\prime}(t)=\frac{1}{2}\left[C(t)+C^{*}(t)\right]
\end{equation}
and an imaginary part
\begin{equation}
C^{\prime\prime}(t)=-\frac{i}{2}\left[C(t)-C^{*}(t)\right]
\end{equation}
so that
\begin{equation}
C(t)=C^{\prime}(t)+iC^{\prime\prime}(t)
\end{equation}



\subsection{Spectral Density}

A very important quantity is the Fourier transform of the bath correlation
function

\begin{equation}
\tilde{C}(\omega)=\int\limits _{0}^{\infty}{\rm d}t\ C(t)e^{i\omega t}=2{\rm Re}\int\limits _{0}^{\infty}{\rm d}t\ C(t)e^{i\omega t}
\end{equation}
It is sometimes referred to as \index{spectral density}\emph{spectral
density}, but we will reserve this name for a different quantity.
We will follow Ref. {[}Mukamel1995{]}. The Fourier transform $\tilde{C}(\omega)$
can be split into even and odd parts defined as
\begin{equation}
\tilde{C}^{\prime}(\omega)=\int\limits _{-\infty}^{\infty}{\rm d}t\ C^{\prime}(t)e^{i\omega t},\;\tilde{C}^{\prime\prime}(\omega)=i\int\limits _{-\infty}^{\infty}{\rm d}t\ C^{\prime\prime}(t)e^{i\omega t}
\end{equation}
so that 
\begin{equation}
\tilde{C}(\omega)=\tilde{C}^{\prime}(\omega)+\tilde{C}^{\prime\prime}(\omega)
\end{equation}
It can be shown (see {[}Mukamel1995{]}) that 
\begin{equation}
\tilde{C}(-\omega)=e^{-\frac{\hbar\omega}{k_{B}T}}\tilde{C}(\omega)\label{eq:C_omega_symmetry}
\end{equation}
 and
\begin{equation}
\tilde{C}(\omega)=\left[1+\coth(\hbar\omega/2k_{B}T)\right]\tilde{C}^{\prime\prime}(\omega)\label{eq:C_omega_by_C_2primes}
\end{equation}
Due to the relation between positive and negative frequency values
of the Fourier transform of the bath correlation function, we can
define it completely through the odd function $\tilde{C}^{\prime\prime}(\omega)$
which is a Fourier transform of the imaginary part of the correlation
function.

Spectral density

\begin{equation}
J(\omega)=\sum_{\xi}|g_{\xi}|^{2}\delta(\omega-\omega_{\xi})
\end{equation}



\section{Standard Redfield Theory}

In this section we define the \index{standard Redfield relaxation tensor}standard
Redfield relaxation tensor in a form which is implemented in Quanta$\rho\epsilon\iota$.
Our formulation is based on the reference {[}MayKuehn{]}


\subsection{General Formula}

\begin{equation}
K_{n}^{(I)}(-\tau)=U_{S}(\tau)K_{n}U_{S}^{\dagger}(\tau)
\end{equation}


\begin{equation}
\Lambda_{m}=\sum_{n}\int\limits _{0}^{\infty}{\rm d}\tau\ C_{mn}(\tau)K_{n}^{(I)}(-\tau)
\end{equation}


\begin{equation}
\frac{\partial}{\partial t}\rho(t)=-\frac{i}{\hbar}\left[H_{S},\rho(t)\right]_{-}-{\cal R}\rho(t)
\end{equation}


\[
{\cal R}\rho(t)=-\frac{1}{\hbar^{2}}\sum_{m}\left(K_{m}\Lambda_{m}\rho(t)+\rho(t)\Lambda_{n}^{\dagger}K_{m}\right)
\]
 

\begin{equation}
+\frac{1}{\hbar^{2}}\sum_{m}\left(K_{m}\rho(t)\Lambda_{m}^{\dagger}+\Lambda_{m}\rho(t)K_{m}\right)
\end{equation}



\subsubsection{Relaxation rate}

In a representation of Hamiltonian eigenstates $|a\rangle$ we can
write e.g.
\[
\frac{\partial}{\partial t}\rho_{aa}(t)=\sum_{b}\left(\frac{1}{\hbar^{2}}\sum_{m}\langle a|K_{m}|b\rangle\rho_{bb}(t)\langle b|\Lambda_{m}^{\dagger}|a\rangle\right)
\]
\begin{equation}
+\sum_{b}\left(\frac{1}{\hbar^{2}}\sum_{m}\langle a|\Lambda_{m}|b\rangle\rho_{bb}(t)\langle b|K_{m}|a\rangle\right),
\end{equation}
where we ignored the so-called non-secular terms which connect oscillating
coherence elements $\rho_{ab}$ of the density matrix with the populations
$\rho_{aa}$. From the definition of $\Lambda_{m}$ we have
\begin{equation}
\langle a|\Lambda_{m}|b\rangle=\sum_{n}\int\limits _{0}^{\infty}{\rm d}\tau\ C_{mn}(\tau)e^{-i\omega_{ab}\tau}\langle a|K_{n}|b\rangle=\sum_{n}\bar{C}_{mn}(\omega_{ba})\langle a|K_{n}|b\rangle,
\end{equation}
where we used that $U_{S}(t)=\sum_{a}e^{-\omega_{a}t}|a\rangle\langle a|$
with $\omega_{a}=\epsilon_{a}/\hbar$ and $\epsilon_{a}$ the eigenergy
of the system corresponding to state $|a\rangle$. For the element
$\langle a|\Lambda_{m}^{\dagger}|b\rangle$ we have $\langle a|\Lambda_{m}^{\dagger}|b\rangle=\langle b|\Lambda_{m}|a\rangle$
and correspondingly the rate of transfer from $|b\rangle$ to $|a\rangle$
is 
\[
\hbar^{2}K_{ab}=\sum_{mn}\bar{C}_{mn}(\omega_{ba})\langle a|K_{n}|b\rangle\langle b|K_{m}|a\rangle+\bar{C}_{mn}^{*}(\omega_{ba})\langle a|K_{m}|b\rangle\langle a|K_{m}|b\rangle
\]
\begin{equation}
\hbar^{2}K_{ab}=\sum_{mn}2{\rm Re}\bar{C}_{mn}(\omega_{ba})\langle a|K_{n}|b\rangle\langle b|K_{m}|a\rangle.
\end{equation}
We can see that the integral above is a half of the Fourier transform
and that there is a relation between the $\bar{C}(\omega)=\int\limits _{0}^{\infty}{\rm d}\tau\ C(\tau)e^{i\omega\tau}$
and the full Fourier transform $\tilde{C}(\omega)=\int\limits _{-\infty}^{\infty}{\rm d}\tau\ C(\tau)e^{i\omega\tau}$
because the correlation function is symmetric
\begin{equation}
C(-t)=C^{*}(t).
\end{equation}
We can show that
\begin{equation}
2{\rm Re}\bar{C}(\omega)=\tilde{C}(\omega),
\end{equation}
and correspondingly
\begin{equation}
\hbar^{2}K_{ab}=\sum_{mn}\tilde{C}_{mn}(\omega_{ba})\langle a|K_{n}|b\rangle\langle b|K_{m}|a\rangle.
\end{equation}



\subsection{Analytical Results}


\subsubsection{Homo-dimer}

The system-bath interaction operators are
\begin{equation}
K_{m}=|m\rangle\langle m|
\end{equation}


If both energy gap fluctuate with the same energy gap correlation
function $C(t)$, and the fluctuations on individual sites are independent
from each other ($C_{mn}(t)=\delta_{mn}C(t))$, the rate $K_{ab}$
of the energy transfer from state $|b\rangle$ to state $|a\rangle$
reads as (we set $\hbar=1$)
\begin{equation}
K_{ab}=\sum_{n}|\langle a|n\rangle|^{2}|\langle b|n\rangle|^{2}\tilde{C}(\omega_{ba}).
\end{equation}
For a homodimer all coefficients $|\langle b|n\rangle|^{2}=\frac{1}{2}$
and the sum over $n$ gives two contributions which are exactly the
same. This means
\begin{equation}
\sum_{n}|\langle a|n\rangle|^{2}|\langle b|n\rangle|^{2}=\sum_{n=1}^{2}\frac{1}{4}=\frac{1}{2}.
\end{equation}
For an overdumped Brownian oscillator spectral density
\begin{equation}
\tilde{C}^{\prime\prime}(\omega)=\frac{2\lambda\omega\left(\frac{1}{\tau_{c}}\right)}{\omega^{2}+\left(\frac{1}{\tau_{c}}\right)^{2}}
\end{equation}
 we have

\begin{equation}
K_{ab}=\frac{1}{2}\tilde{C}(\omega_{ba})=\frac{1}{2}\left(1+\coth\left(\frac{\omega_{ba}}{2k_{B}T}\right)\right)\frac{2\lambda\omega_{ba}\left(\frac{1}{\tau_{c}}\right)}{\omega_{ba}^{2}+\left(\frac{1}{\tau_{c}}\right)^{2}}.
\end{equation}
Given that for a homodimer $\omega_{12}=2J$ we have
\begin{equation}
K_{12}=\frac{1}{2}\tilde{C}(\omega_{21})=\frac{\lambda J}{2}\frac{1+\coth\left(\frac{J}{k_{B}T}\right)}{J{}^{2}\tau_{c}+\frac{\hbar^{2}}{4\tau_{c}}}.
\end{equation}


This formula is used to test the calculations of Redfield rates and
of the Redfield tensor in Quanta$\rho\epsilon\iota$.


\subsubsection{Hetero-dimer}

\appendix


\section{Microscopic Derivation of Spectral Density Symmetries}

Here we will derive the Eqs. (\ref{eq:C_omega_symmetry}) and (\ref{eq:C_omega_by_C_2primes}).
\end{document}
